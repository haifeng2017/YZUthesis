% !TEX encoding = UTF-8

%如果从第0章开始, 则修改计数器 chapter 为 -1
\setcounter{chapter}{-1}

\chapter{前言}

YZUthesis 模板最初完成于作者攻读博士学位期间, 基于 CASthesis\acknowledge{CASthesis 是中国科学院硕士、博士论文LaTeX模板, 作者吴凌云. 模板可在github上下载. \href{https://github.com/huashi1/CAS_Thesis_Latex} {https://github.com/huashi1/CAS\_Thesis\_Latex}.}, 其中导师签字页的设计, 则参考了加拿大达尔豪斯大学(Dalhousie University)的学位论文模板.

历经多年, 原模板已难以适配最新 TeXLive 系统的编译环境, 更新工作迫在眉睫. 本次修订基本遵循《扬州大学研究生学位论文格式规范(试行)》(校研院〔2020〕4号)(简称“论文格式规范文件”)的要求, 力求在格式上基本符合学校规定. 我们希望这一更新能为研究生们提供切实的便利, 使其从繁琐的排版调整工作中解脱出来, 将宝贵的时间与精力投入于论文写作与打磨中.



\section{学位论文的格式规范}

这里列举一些“论文格式规范文件”中关于论文格式规范需要注意的地方. 该文件规定了论文页面的具体设置\footnote{版芯要求: 左边距: 26mm, 右边距: 26mm, 上边距: 30mm, 下边距: 25mm. 页眉距边界20mm, 页脚距边界17.5mm.}. 学位论文由下列部分组成并按此顺序排列.
\begin{itemize}
  \item (1)封面;
  \item (2)英文扉页;
  \item (3)基金资助页;
  \item (4)论文原创性声明和版权使用授权书;
  \item (5)目录;
  \item (6)中文摘要及关键词;
  \item (7)英文摘要及关键词;
  \item (8)符号说明;
  \item (9)前言(绪言或绪论);
  \item (10)正文;
  \item (11)结语;
  \item (12)参考文献;
  \item (13)附录;
  \item (14)攻读学位期间取得的研究成果;
  \item (15)致谢.
\end{itemize}

不过本模板将参考文献放在附录的后面.







