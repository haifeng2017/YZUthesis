% !TEX encoding = UTF-8
% 符号说明页
% 解释论文中所用符号表示的意义、单位或量纲等.
% 符号说明主要为帮助阅读论文, 如论文中没有采用符号则该部分略去.
\begin{symbols}


\section*{微分几何常用符号说明}



\begin{table}[htbp]
    \centering
    \caption{微分几何常用符号表}
    \label{tab:symbols}
    \begin{tabularx}{\textwidth}{l>{\raggedright\arraybackslash}Xl}
        \toprule
        \textbf{符号} & \textbf{含义} & \textbf{备注} \\
        \midrule
        $M$, $N$ & 光滑流形 & 通常为有限维 \\
        $\mathcal{F}(M)$ & $M$ 上光滑函数环 & \\
        $T_p M$ & 流形 $M$ 在点 $p$ 处的切空间 & \\
        $T M$ & 流形 $M$ 的切丛 & \\
        $\mathfrak{X}(M)$ & $M$ 上的光滑向量场全体 & \\
        $\omega$, $\eta$ & 微分形式 & \\
        $\Omega^k(M)$ & $M$ 上的 $k$-次微分形式空间 & \\
        $d$ & 外微分算子 & $d: \Omega^k(M) \to \Omega^{k+1}(M)$ \\
        $\wedge$ & 外积(楔积) & \\
        $\mathcal{L}_X$ & 关于向量场 $X$ 的李导数 & \\
        $[X, Y]$ & 向量场 $X$ 与 $Y$ 的李括号 & \\
        $\nabla$ & 仿射联络(或协变导数) & \\
        $\nabla_X Y$ & 向量场 $Y$ 沿 $X$ 方向的协变导数 & \\
        $\Gamma_{ij}^k$ & Christoffel 符号 & 对于坐标基 $\partial_i$ \\
        $g$, $\langle \cdot, \cdot \rangle$ & 黎曼度量 & \\
        $R$ & 曲率张量 & $R(X,Y)Z = \nabla_X \nabla_Y Z - \nabla_Y \nabla_X Z - \nabla_{[X,Y]} Z$ \\
        $R_{ijk}^{\ \ \ l}$ & 曲率张量的坐标分量 & \\
        $\mathrm{Ric}$ & Ricci 曲率张量 & \\
        $S$ & 标量曲率 & \\
        $\exp_p$ & 在点 $p$ 处的指数映射 & \\
        $\mathrm{inj}(p)$ & 点 $p$ 处的单射半径 & \\
        $\mathrm{Hess}(f)$ & 函数 $f$ 的 Hessian & 在黎曼流形上 $\mathrm{Hess}(f) = \nabla^2 f$ \\
        $\Delta$ & Laplace–Beltrami 算子 & \\
        $\partial M$ & 流形 $M$ 的边界 & \\
        $\nu$ & 边界上的单位外法向量场 & \\
        $\mathbb{S}^n$ & $n$ 维球面 & \\
        $\mathbb{RP}^n$ & $n$ 维实射影空间 & \\
        $\mathbb{CP}^n$ & $n$ 维复射影空间 & \\
        \bottomrule
    \end{tabularx}
\end{table}

\end{symbols}

