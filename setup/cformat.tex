% !TEX encoding = UTF-8
%%====<<宏包>>============================================================
% 已包含的宏包 |
%--------------|
%%中文TeX宏包
%%{ctex}
%% math packages
%%{amsmath,amsthm,amsfonts,amssymb,bm,xypic,dsfont,pifont}
%% graphics packages
%%{graphicx,psfrag}
%% color package
%%{color}
%% makeidx
%%{makeidx}
%使用宏包tocbibind可以将参考文献或索引等放置到目录中去, 需要stdclsdv宏包
%%{stdclsdv,tocbibind}
%% 一些特殊符号
%%{bbding}
%% check pdfTeX mode
%%{ifpdf}
%%交叉索引
%%{hyperref}
%--------------------------|
% 加入你需要用到的其他宏包 |
%--------------------------|
%\usepackage{你需要用到的宏包名称}
%------------------------------------|
% 下面是一种数学字体, 详见symbols.pdf|
%------------------------------------|
\DeclareMathAlphabet{\mathpzc}{OT1}{pzc}{m}{it}
%用法: \eq{\mathpzc{f(x)}}
%--*--<>--*--<>--*--<>--*--<>--*--<>--*--<>--*--<>--*--<>--*--<>--*--

%====<<格式>>============================================================
% THEOREMS -------------------------------------------------------
\newtheorem{thm}{定理}[section]
\newtheorem{cor}[thm]{推论}
\newtheorem{lem}[thm]{引理}
\newtheorem{prop}[thm]{命题}
\theoremstyle{definition}
\newtheorem{defn}[thm]{定义}
\newtheorem{conj}{猜想~}
\newtheorem{exmp}{例~}
\newtheorem{exer}{习题~}
%\theoremstyle{remark}
\newtheorem{rem}[thm]{注}
\numberwithin{equation}{section}
\renewcommand{\proofname}{\bf 证明}
% MATH -----------------------------------------------------------
%范数, 绝对值, 集合, 内积
\newcommand{\norm}[1]{\left\Vert#1\right\Vert}  %%范数, 用法$\norm{f}$ 即 ||f||
\newcommand{\abs}[1]{\left\vert#1\right\vert}   %%绝对值, 用法$\abs{f}$ 即 |f|
\newcommand{\set}[1]{\left\{#1\right\}}         %%集合, 用法$\set{x:\frac{1}{x}<1}$ 即{x:1/x<1}
\newcommand{\seq}[1]{\left<#1\right>}
\newcommand{\sseq}[1]{\left<\left<#1\right>\right>}
\newcommand{\essnorm}[1]{\norm{#1}_{\text{\rm\normalshape ess}}}
%-----------------------------------------------------------------
%各种符号简记:实数R, varepsilon,
\newcommand{\Real}{\mathbb R}                   %%实数集合, 花写的 R 直接用 $\Real$
\newcommand{\Complex}{\mathbb C}                %%复数集合
\newcommand{\Integer}{\mathbb Z}                %%整数集合
\newcommand{\eps}{\varepsilon}                  %%\eps 代替复杂的 \varepsilon
\newcommand{\dd}{{\cal{D}}}
\newcommand{\pp}{\partial}
\newcommand{\BX}{\mathbf{B}(X)}                 %% 个人可以设置
\newcommand{\A}{\mathcal{A}}                    %% 看自己的喜好
\newcommand{\rr}{{\cal R}}
\newcommand{\minvol}{\textrm{MinVol}}
\newcommand{\vol}{\textrm{Vol}}
\newcommand{\simplvol}{\textrm{SimplVol}}
\newcommand{\const}{\textrm{const}}
\newcommand{\rt}{\rightarrow}
\newcommand{\lt}{\leftarrow}
\newcommand{\xrt}{\xrightarrow}
\newcommand{\xlt}{\xleftarrow}
\newcommand{\Int}{\textrm{Int}}
\newcommand{\Hyperbolic}{\mathbb H}
\newcommand{\Ric}{\textrm{Ricci}}
\newcommand{\htop}{\textrm{h}_{\textrm{top}}}
\newcommand{\h}{{\textrm h}}
\newcommand{\im}{\textrm{Im}}
\newcommand{\Ker}{\textrm{Ker}}
\newcommand{\z}[1]{~#1~}
\newcommand{\eq}[1]{~$#1$~}  %%使用时如果后面紧跟标点符号, 就将标点符号放在里面, 这样避免标点符号出现在行首.
\newcommand{\Ad}{\textrm{Ad}}
\newcommand{\id}{\mathds{1}}
\newcommand{\C}{{\cal C}}  % 用于联络空间C(P)
%------------------------------------------------------------------
%映射, 实部, 虚部
\newcommand{\To}{\longrightarrow}               %%\To  即产生长的右向箭头---->
\newcommand{\RE}{\operatorname{Re}}             %% 实部
\newcommand{\IM}{\operatorname{Im}}             %% 虚部
%\newcommand{\Poly}{{\cal{P}}(E)}
%\newcommand{\EssD}{{\cal{D}}}
% ----------------------------------------------------------------
\newcommand{\ds}{\displaystyle}                 %% \ds 代替复杂的 \displaystyle
% ----------------------------------------------------------------
% Font
\newcommand{\song}{\CJKfamily{song}}    % 宋体   (Windows自带simsun.ttf)
\newcommand{\fs}{\CJKfamily{fs}}        % 仿宋体 (Windows自带simfs.ttf)
\newcommand{\kai}{\CJKfamily{kai}}      % 楷体   (Windows自带simkai.ttf)
\newcommand{\hei}{\CJKfamily{hei}}      % 黑体   (Windows自带simhei.ttf)
\newcommand{\li}{\CJKfamily{li}}        % 隶书   (Windows自带simli.ttf)
%---------------------------------------------------------------------
% Color
%\definecolor{mygreen}{rgb}{0,0.5,0}
%\newcommand{\tc}[1]{\textcolor{mygreen}{#1}\index{#1}}

\newenvironment{mytip}{\texttt{MyTip}\ding{46}\zihao{-5}\kaishu}{\ding{122}}

%%用于中文环境中, 以增加适度的空格
\newcommand{\Cite}[1]{~\cite{#1}~}
\newcommand{\kh}[1]{~({#1})~}%%括号

%====================================================================
% 希望有更多更好看的装饰条
%  带说明
%====<<>>============================================================
% 不带说明
%--*--<>--*--<>--*--<>--*--<>--*--<>--*--<>--*--<>--*--<>--*--<>--*--
