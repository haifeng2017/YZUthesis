% !TEX encoding = UTF-8
%%====<<宏包>>============================================================
% 已包含的宏包 |
%--------------|
%%中文TeX宏包
%%{ctex}
%% math packages
%%{amsmath,amsthm,amsfonts,amssymb,bm,xypic,dsfont,pifont}
%% graphics packages
%%{graphicx,psfrag}
%% color package
%%{color}
%% makeidx
%%{makeidx}
%使用宏包tocbibind可以将参考文献或索引等放置到目录中去,需要stdclsdv宏包
%%{stdclsdv,tocbibind}
%% 一些特殊符号
%%{bbding}
%% check pdfTeX mode
%%{ifpdf}
%%交叉索引
%%{hyperref}
%--------------------------|
% 加入你需要用到的其他宏包 |
%--------------------------|
%\usepackage{你需要用到的宏包名称}
\usepackage{mathptmx}
%------------------------------------|
% 下面是一种数学字体, 详见symbols.pdf|
%------------------------------------|
\DeclareMathAlphabet{\mathpzc}{OT1}{pzc}{m}{it}
%用法: \eq{\mathpzc{f(x)}}
%--*--<>--*--<>--*--<>--*--<>--*--<>--*--<>--*--<>--*--<>--*--<>--*--

%====<<格式>>============================================================
% THEOREMS -------------------------------------------------------
\newtheorem{thm}{Theorem}[chapter]
\newtheorem{cor}[thm]{Corollary}
\newtheorem{lem}[thm]{Lemma}
\newtheorem{prop}[thm]{Proposition}
\theoremstyle{definition}
\newtheorem{defn}[thm]{Definition}
\theoremstyle{remark}
\newtheorem{rem}[thm]{Remark}
\numberwithin{equation}{section}
% MATH -----------------------------------------------------------
%范数,绝对值,集合,内积
\newcommand{\norm}[1]{\left\Vert#1\right\Vert}  %%范数,用法$\norm{f}$ 即 ||f||
\newcommand{\abs}[1]{\left\vert#1\right\vert}   %%绝对值,用法$\abs{f}$ 即 |f|
\newcommand{\set}[1]{\left\{#1\right\}}         %%集合,用法$\set{x:\frac{1}{x}<1}$ 即{x:1/x<1}
\newcommand{\seq}[1]{\left<#1\right>}
\newcommand{\sseq}[1]{\left<\left<#1\right>\right>}
\newcommand{\essnorm}[1]{\norm{#1}_{\text{\rm\normalshape ess}}}
%-----------------------------------------------------------------
%各种符号简记:实数R,varepsilon,
\newcommand{\Real}{\mathbb R}                   %%实数集合,花写的 R 直接用 $\Real$
\newcommand{\Complex}{\mathbb C}                %%复数集合
\newcommand{\Integer}{\mathbb Z}                %%整数集合
\newcommand{\eps}{\varepsilon}                  %%\eps 代替复杂的 \varepsilon
\newcommand{\dd}{{\cal{D}}}
\newcommand{\pp}{\partial}
%------------------------------------------------------------------
%映射,实部,虚部
\newcommand{\To}{\longrightarrow}               %%\To  即产生长的右向箭头---->
\newcommand{\RE}{\operatorname{Re}}             %% 实部
\newcommand{\IM}{\operatorname{Im}}             %% 虚部
%\newcommand{\Poly}{{\cal{P}}(E)}
%\newcommand{\EssD}{{\cal{D}}}
% ----------------------------------------------------------------
\newcommand{\ds}{\displaystyle}                 %% \ds 代替复杂的 \displaystyle
% ----------------------------------------------------------------
%个人自定义设置
\newcommand{\const}{\textrm{const}}
\newcommand{\rt}{\rightarrow}
\newcommand{\lt}{\leftarrow}
\newcommand{\xrt}{\xrightarrow}
\newcommand{\xlt}{\xleftarrow}
\newcommand{\Int}{\textrm{Int}}
\newcommand{\h}{{\textrm h}}
\newcommand{\im}{\textrm{Im}}
\newcommand{\Ker}{\textrm{Ker}}
\newcommand{\Ad}{\textrm{Ad}}
\newcommand{\id}{\mathds{1}}
%下面两个命令用于中文环境中
\newcommand{\z}[1]{~#1~}
\newcommand{\eq}[1]{~$#1$~}  %%使用时如果后面紧跟标点符号, 就将标点符号放在里面, 这样避免标点符号出现在行首.
%---------------------------------------------------------------------
% Color
% 已定义下面的\tc命令, 用于将关键字加色并添加到索引中.
%\definecolor{mygreen}{rgb}{0,0.5,0}
%\newcommand{\tc}[1]{\textcolor{mygreen}{#1}\index{#1}}
%====================================================================
% 希望有更多更好看的装饰条
%  带说明
%====<<>>============================================================
% 不带说明
%--*--<>--*--<>--*--<>--*--<>--*--<>--*--<>--*--<>--*--<>--*--<>--*--
