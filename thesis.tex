% !TEX encoding = UTF-8
%%%%%%%%%%%%%%%%%%%%%%%%%%%%%%%%%%%%%%%%%%%%%%%%%%%%%%%%%%%%%%%%%%%%%%
%% Thesis Template of Yangzhou University
%%   ver 1.0 created on 2007-3-28 by H. Xu
%%   modified from:
%%
%% Thesis Template of Chinese Academy of Sciences
%%   for using CASthesis package with LaTeX2e
%%
%% Created by Ling-Yun Wu <aloft@ctex.org>
%%
%% $Id: template.tex,v 1.8 2006/09/09 08:50:46 aloft Exp $
%%%%%%%%%%%%%%%%%%%%%%%%%%%%%%%%%%%%%%%%%%%%%%%%%%%%%%%%%%%%%%%%%%%%%%
%%  Version 2.0 created on 2026-1-30
%%                                 by H. Xu
%%********************************************************************
%|======================|%
%|    学位论文模板                    |%
%|======================|%
%
% 特点:
%     1. 适合撰写博士, 硕士, 本科毕业论文.
%     2. 稍作修改可以适合其他学校.
%                                                         --- H. Xu
%%********************************************************************


%%<*>--<*>--<*>--<*>--<*>--<*>--<*>--<*>--<*>--<*>--<*>--<*>--<*>--<*>
%
\documentclass[doctor]{YZUthesis}%博士学位论文
%\documentclass[master]{YZUthesis}%硕士学位论文
%\documentclass[bachelor]{YZUthesis}%学士学位论文
%%--------------------------------------------------------------------
%\documentclass[dvipdfm]{YZUthesis}%适用 LaTeX编译, 然后运行 dvipdfmx 将dvi转换为pdf.
%\documentclass[dvipdfm,nocap,noindent]{YZUthesis}%
% 可选参数:
% notypeinfo 取消扉页的LaTeX版本信息
% dvips 使用 dvips 生成最终的 PS 文档
% dvipdfm 使用 dvipdfm(x) 生成最终的 PDF 文档
% nocap 表示保留英文标题格式
% noindent 英文格式, 章节标题后的段首不缩进
%%<*>--<*>--<*>--<*>--<*>--<*>--<*>--<*>--<*>--<*>--<*>--<*>--<*>--<*>


%%<*>--<*>--<*>--<*>--<*>--<*>--<*>--<*>--<*>--<*>--<*>--<*>--<*>--<*>
%|============|%
%|  一些设置  |%
%|============|%
% 设置图形文件的搜索路径
\graphicspath{{fig/}}
%
% 要取消链接的颜色(黑白打印时), 请使用下面两个命令.
% 但是实际表明打印出来的黑白效果不佳, 电脑上显示还可以.
%\hypersetup{colorlinks=false}
%\renewcommand{\tc}[1]{\emph{#1}\index{\emph{#1}}}
%------------------------------------------------------


% 小节标题靠左对齐
%\CTEXsetup[format+={\flushleft}]{section}
%%<*>--<*>--<*>--<*>--<*>--<*>--<*>--<*>--<*>--<*>--<*>--<*>--<*>--<*>


%%<*>--<*>--<*>--<*>--<*>--<*>--<*>--<*>--<*>--<*>--<*>--<*>--<*>--<*>
% 可以修改以下两个文件, 以设定自己的格式.
% !TEX encoding = UTF-8
%%====<<宏包>>============================================================
% 已包含的宏包 |
%--------------|
%%中文TeX宏包
%%{ctex}
%% math packages
%%{amsmath,amsthm,amsfonts,amssymb,bm,xypic,dsfont,pifont}
%% graphics packages
%%{graphicx,psfrag}
%% color package
%%{color}
%% makeidx
%%{makeidx}
%使用宏包tocbibind可以将参考文献或索引等放置到目录中去, 需要stdclsdv宏包
%%{stdclsdv,tocbibind}
%% 一些特殊符号
%%{bbding}
%% check pdfTeX mode
%%{ifpdf}
%%交叉索引
%%{hyperref}
%--------------------------|
% 加入你需要用到的其他宏包 |
%--------------------------|
%\usepackage{你需要用到的宏包名称}
%------------------------------------|
% 下面是一种数学字体, 详见symbols.pdf|
%------------------------------------|
\DeclareMathAlphabet{\mathpzc}{OT1}{pzc}{m}{it}
%用法: \eq{\mathpzc{f(x)}}
%--*--<>--*--<>--*--<>--*--<>--*--<>--*--<>--*--<>--*--<>--*--<>--*--

%====<<格式>>============================================================
% THEOREMS -------------------------------------------------------
\newtheorem{thm}{定理}[section]
\newtheorem{cor}[thm]{推论}
\newtheorem{lem}[thm]{引理}
\newtheorem{prop}[thm]{命题}
\theoremstyle{definition}
\newtheorem{defn}[thm]{定义}
\newtheorem{conj}{猜想~}
\newtheorem{exmp}{例~}
\newtheorem{exer}{习题~}
%\theoremstyle{remark}
\newtheorem{rem}[thm]{注}
\numberwithin{equation}{section}
\renewcommand{\proofname}{\bf 证明}
% MATH -----------------------------------------------------------
%范数, 绝对值, 集合, 内积
\newcommand{\norm}[1]{\left\Vert#1\right\Vert}  %%范数, 用法$\norm{f}$ 即 ||f||
\newcommand{\abs}[1]{\left\vert#1\right\vert}   %%绝对值, 用法$\abs{f}$ 即 |f|
\newcommand{\set}[1]{\left\{#1\right\}}         %%集合, 用法$\set{x:\frac{1}{x}<1}$ 即{x:1/x<1}
\newcommand{\seq}[1]{\left<#1\right>}
\newcommand{\sseq}[1]{\left<\left<#1\right>\right>}
\newcommand{\essnorm}[1]{\norm{#1}_{\text{\rm\normalshape ess}}}
%-----------------------------------------------------------------
%各种符号简记:实数R, varepsilon,
\newcommand{\Real}{\mathbb R}                   %%实数集合, 花写的 R 直接用 $\Real$
\newcommand{\Complex}{\mathbb C}                %%复数集合
\newcommand{\Integer}{\mathbb Z}                %%整数集合
\newcommand{\eps}{\varepsilon}                  %%\eps 代替复杂的 \varepsilon
\newcommand{\dd}{{\cal{D}}}
\newcommand{\pp}{\partial}
\newcommand{\BX}{\mathbf{B}(X)}                 %% 个人可以设置
\newcommand{\A}{\mathcal{A}}                    %% 看自己的喜好
\newcommand{\rr}{{\cal R}}
\newcommand{\minvol}{\textrm{MinVol}}
\newcommand{\vol}{\textrm{Vol}}
\newcommand{\simplvol}{\textrm{SimplVol}}
\newcommand{\const}{\textrm{const}}
\newcommand{\rt}{\rightarrow}
\newcommand{\lt}{\leftarrow}
\newcommand{\xrt}{\xrightarrow}
\newcommand{\xlt}{\xleftarrow}
\newcommand{\Int}{\textrm{Int}}
\newcommand{\Hyperbolic}{\mathbb H}
\newcommand{\Ric}{\textrm{Ricci}}
\newcommand{\htop}{\textrm{h}_{\textrm{top}}}
\newcommand{\h}{{\textrm h}}
\newcommand{\im}{\textrm{Im}}
\newcommand{\Ker}{\textrm{Ker}}
\newcommand{\z}[1]{~#1~}
\newcommand{\eq}[1]{~$#1$~}  %%使用时如果后面紧跟标点符号, 就将标点符号放在里面, 这样避免标点符号出现在行首.
\newcommand{\Ad}{\textrm{Ad}}
\newcommand{\id}{\mathds{1}}
\newcommand{\C}{{\cal C}}  % 用于联络空间C(P)
%------------------------------------------------------------------
%映射, 实部, 虚部
\newcommand{\To}{\longrightarrow}               %%\To  即产生长的右向箭头---->
\newcommand{\RE}{\operatorname{Re}}             %% 实部
\newcommand{\IM}{\operatorname{Im}}             %% 虚部
%\newcommand{\Poly}{{\cal{P}}(E)}
%\newcommand{\EssD}{{\cal{D}}}
% ----------------------------------------------------------------
\newcommand{\ds}{\displaystyle}                 %% \ds 代替复杂的 \displaystyle
% ----------------------------------------------------------------
% Font
\newcommand{\song}{\CJKfamily{song}}    % 宋体   (Windows自带simsun.ttf)
\newcommand{\fs}{\CJKfamily{fs}}        % 仿宋体 (Windows自带simfs.ttf)
\newcommand{\kai}{\CJKfamily{kai}}      % 楷体   (Windows自带simkai.ttf)
\newcommand{\hei}{\CJKfamily{hei}}      % 黑体   (Windows自带simhei.ttf)
\newcommand{\li}{\CJKfamily{li}}        % 隶书   (Windows自带simli.ttf)
%---------------------------------------------------------------------
% Color
%\definecolor{mygreen}{rgb}{0,0.5,0}
%\newcommand{\tc}[1]{\textcolor{mygreen}{#1}\index{#1}}

\newenvironment{mytip}{\texttt{MyTip}\ding{46}\zihao{-5}\kaishu}{\ding{122}}

%%用于中文环境中, 以增加适度的空格
\newcommand{\Cite}[1]{~\cite{#1}~}
\newcommand{\kh}[1]{~({#1})~}%%括号

%====================================================================
% 希望有更多更好看的装饰条
%  带说明
%====<<>>============================================================
% 不带说明
%--*--<>--*--<>--*--<>--*--<>--*--<>--*--<>--*--<>--*--<>--*--<>--*--
 %%中文风格, 适用于论文都是用中文写的
%% !TEX encoding = UTF-8
%%====<<宏包>>============================================================
% 已包含的宏包 |
%--------------|
%%中文TeX宏包
%%{ctex}
%% math packages
%%{amsmath,amsthm,amsfonts,amssymb,bm,xypic,dsfont,pifont}
%% graphics packages
%%{graphicx,psfrag}
%% color package
%%{color}
%% makeidx
%%{makeidx}
%使用宏包tocbibind可以将参考文献或索引等放置到目录中去,需要stdclsdv宏包
%%{stdclsdv,tocbibind}
%% 一些特殊符号
%%{bbding}
%% check pdfTeX mode
%%{ifpdf}
%%交叉索引
%%{hyperref}
%--------------------------|
% 加入你需要用到的其他宏包 |
%--------------------------|
%\usepackage{你需要用到的宏包名称}
\usepackage{mathptmx}
%------------------------------------|
% 下面是一种数学字体, 详见symbols.pdf|
%------------------------------------|
\DeclareMathAlphabet{\mathpzc}{OT1}{pzc}{m}{it}
%用法: \eq{\mathpzc{f(x)}}
%--*--<>--*--<>--*--<>--*--<>--*--<>--*--<>--*--<>--*--<>--*--<>--*--

%====<<格式>>============================================================
% THEOREMS -------------------------------------------------------
\newtheorem{thm}{Theorem}[chapter]
\newtheorem{cor}[thm]{Corollary}
\newtheorem{lem}[thm]{Lemma}
\newtheorem{prop}[thm]{Proposition}
\theoremstyle{definition}
\newtheorem{defn}[thm]{Definition}
\theoremstyle{remark}
\newtheorem{rem}[thm]{Remark}
\numberwithin{equation}{section}
% MATH -----------------------------------------------------------
%范数,绝对值,集合,内积
\newcommand{\norm}[1]{\left\Vert#1\right\Vert}  %%范数,用法$\norm{f}$ 即 ||f||
\newcommand{\abs}[1]{\left\vert#1\right\vert}   %%绝对值,用法$\abs{f}$ 即 |f|
\newcommand{\set}[1]{\left\{#1\right\}}         %%集合,用法$\set{x:\frac{1}{x}<1}$ 即{x:1/x<1}
\newcommand{\seq}[1]{\left<#1\right>}
\newcommand{\sseq}[1]{\left<\left<#1\right>\right>}
\newcommand{\essnorm}[1]{\norm{#1}_{\text{\rm\normalshape ess}}}
%-----------------------------------------------------------------
%各种符号简记:实数R,varepsilon,
\newcommand{\Real}{\mathbb R}                   %%实数集合,花写的 R 直接用 $\Real$
\newcommand{\Complex}{\mathbb C}                %%复数集合
\newcommand{\Integer}{\mathbb Z}                %%整数集合
\newcommand{\eps}{\varepsilon}                  %%\eps 代替复杂的 \varepsilon
\newcommand{\dd}{{\cal{D}}}
\newcommand{\pp}{\partial}
%------------------------------------------------------------------
%映射,实部,虚部
\newcommand{\To}{\longrightarrow}               %%\To  即产生长的右向箭头---->
\newcommand{\RE}{\operatorname{Re}}             %% 实部
\newcommand{\IM}{\operatorname{Im}}             %% 虚部
%\newcommand{\Poly}{{\cal{P}}(E)}
%\newcommand{\EssD}{{\cal{D}}}
% ----------------------------------------------------------------
\newcommand{\ds}{\displaystyle}                 %% \ds 代替复杂的 \displaystyle
% ----------------------------------------------------------------
%个人自定义设置
\newcommand{\const}{\textrm{const}}
\newcommand{\rt}{\rightarrow}
\newcommand{\lt}{\leftarrow}
\newcommand{\xrt}{\xrightarrow}
\newcommand{\xlt}{\xleftarrow}
\newcommand{\Int}{\textrm{Int}}
\newcommand{\h}{{\textrm h}}
\newcommand{\im}{\textrm{Im}}
\newcommand{\Ker}{\textrm{Ker}}
\newcommand{\Ad}{\textrm{Ad}}
\newcommand{\id}{\mathds{1}}
%下面两个命令用于中文环境中
\newcommand{\z}[1]{~#1~}
\newcommand{\eq}[1]{~$#1$~}  %%使用时如果后面紧跟标点符号, 就将标点符号放在里面, 这样避免标点符号出现在行首.
%---------------------------------------------------------------------
% Color
% 已定义下面的\tc命令, 用于将关键字加色并添加到索引中.
%\definecolor{mygreen}{rgb}{0,0.5,0}
%\newcommand{\tc}[1]{\textcolor{mygreen}{#1}\index{#1}}
%====================================================================
% 希望有更多更好看的装饰条
%  带说明
%====<<>>============================================================
% 不带说明
%--*--<>--*--<>--*--<>--*--<>--*--<>--*--<>--*--<>--*--<>--*--<>--*--
 %%英文风格, 适用于论文主体部分都是英文写的
%%<*>--<*>--<*>--<*>--<*>--<*>--<*>--<*>--<*>--<*>--<*>--<*>--<*>--<*>

\begin{document}

%%--------------------------------------------------------------------
%|===========|%
%| 封面顶部  |%
%|===========|%

  % 请输入中文封面内容, 如不确定或不需要填, 则不必输入
  \classification{}                                        %分类号
  \serialnumber{D1234}                                     %学号
  \UDC{}                                                   %UDC
  \confidential{}                                          %密级

%%--------------------------------------------------------------------
%|===========|%
%| 论文标题        |%
%|===========|%

  %\title[]{} 命令用于填写论文标题, 其中[]内的用于显示在页眉, 故不宜太长.
  %如果标题太长, 请在适当的位置使用\bigskip
  %当然\smallskip,\medskip,\goodbreak 都可以用, 但是这里\bigskip的间隔比较好.
  %例如下面的长标题, 可以试一下
  %\title[某某某\quad 微分几何~Differential Geometry~和偏微分方程~Partial Differential Equation~的联系]
  %{微分几何~Differential Geometry, 代数几何~Algebraic Geometry, 代数拓扑~Algebraic Topology~和偏微分方程~Partial Differential Equation~的联系~abcdefghijklmnopqrstuvwxyz
  %\uppercase{abcdefghijklmnopqrstuvwxyz}abcdefghijklmnopqrstuvwxyz}
  \title[某某某\quad 几何与拓扑]{几何与拓扑}

  %不论你的题目是不是很长, 请再拷贝一下到\inlinetitle, 此时请不要加\bigskip, 如下面的.
  %注意论文标题太长会影响版面, 最好不要超过五行.
  %\inlinetitle{微分几何~Differential Geometry, 代数几何~Algebraic Geometry, 代数拓扑~Algebraic Topology~和偏微分方程~Partial Differential Equation~的联系~abcdefghijklmnopqrstuvwxyz
  %\uppercase{abcdefghijklmnopqrstuvwxyz}abcdefghijklmnopqrstuvwxyz}
  \inlinetitle{几何与拓扑}
%%--------------------------------------------------------------------
%|====================|%
%| 请认真填写基本信息 |%
%|====================|%

  %\singlename          %如不是单名, 请注释掉该命令
  \author{某\ 某\ 某}    %作者, 显示于首页. 单名的话中间最好加\quad, 双名的话每个字中间加"\ "比较好看.
  \inlineauthor{某某某}  %同author, 中间没有空格, 用于正文中.
  %\degree{博\ 士}        %申请学位级别, 显示在首页, 中间加\quad是为了好看
  %\inlinedegree{博士}    %输入“博士”, 显示在其他地方, 中间不要加空格
  \major{微分几何}        %学科专业名称
  \submitdate{2026年5月} %论文提交日期
  \defenddate{2026年6月} %论文答辩日期
  \DegreeGrantedDate{}  %学位授予日期
  \dept{数学学院}         %作者所在院系
  \university{扬州大学}   %学位授予单位
  \supervisor{A~教授}     %导师
  \supervisordept{扬州大学,\quad 江苏扬州,\quad 225002} %导师所在研究机构, 地点, 邮编
  \cosupervisor{B~教授}                               %副导师, 如没有请加注释
  \cosupervisordept{某某大学,\quad 某省某地,\quad 210000}%副导师所在研究机构, 如没有请加注释
  \funding{国家自然科学基金}%基金项目名称及项目号等信息
  \chairman{   }         %答辩委员会主席
  \firstreader{   }      %论文第一评阅人
  \secondreader{   }     %论文第二评阅人
  \thirdreader{   }      %论文第三评阅人
  \fourthreader{   }     %论文第四评阅人, 如没有请加注释
  \fifthreader{   }      %论文第五评阅人, 如没有请加注释
  %%本模版最多提供五位评阅人, 如没有第三、四、五评阅人请加上注释
  %%博士学位论文一般至少三位评阅人.

%%--------------------------------------------------------------------
%|=====================|%
%| 请输入英文封面内容  |%
%|=====================|%

  % 请输入英文封面内容, 如不确定或不需要填, 则不必输入
  \englishtitle{Geometry and Topology}
  %下面的标题太长了, 最好不要超过五行.
  %\englishtitle{微分几何~Differential Geometry, 代数几何~Algebraic Geometry, 代数拓扑~Algebraic Topology~和偏微分方程~Partial Differential Equation~的联系~abcdefghijklmnopqrstuvwxyz
  %\uppercase{abcdefghijklmnopqrstuvwxyz}abcdefghijklmnopqrstuvwxyz}
  \englishauthor{Your Name}
  \englishsupervisor{Prof. A}
  \englishcosupervisor{Associate Prof. B}
  \englishdept{School of Mathematical Science} %作者所在院系
  \englishuniversity{Yangzhou University} %作者所在学校
  \englishdegree{Ph.D.}
  %\englishmajor{Differential Geometry}
  \englishmajor{Science}


%%--------------------------------------------------------------------
%|=================================|%
%| 生成各封面, 如不需要可将其注释  |%
%|=================================|%

  % 生成中文封面
  \maketitle

  % 生成英文封面
  \makeenglishtitle

  % 基金资助页
  % !TEX encoding = UTF-8
% 基金资助页
\begin{NSFC}

%\fundingnote

本论文研究得到国家自然科学基金(编号: ) 的资助.
\end{NSFC}



  % 论文原创性声明和版权使用授权书
  \statementpage

  % 生成导师、答辩委员会主席、论文评审人签字页面.
  \supervisorsignature

%%--------------------------------------------------------------------
%|==========================================|%
%| 前言部分, 包括中英文摘要、目录、符号说明 |%
%|==========================================|%
\frontmatter  % 前置部分开始

% 目录、摘要、符号说明等

  % 目录
  \tableofcontents
  % 表格目录
  % \listoftables
  % 插图目录
  % \listoffigures

  % 摘要
  %%注意不需要加.tex, 否则可能不能用\include命令, 对于以下的也一样
  % !TEX encoding = UTF-8
% 中英文摘要


% 中文摘要
% 一般要求博士生不少于 1500 字,硕士生不少于 600 字。
% 论文摘要一般包括:论文工作的目的和意义、研究方法、研究成果和新的见解。
% 论文关键词 3~5 个,关键词之间用分号分开,最后一个关键词后不打标点符号。
\begin{abstract}
本文主要工作是 ...


\keywords{关键词1; 关键词2; 关键词3} % 关键词 3~5个
\end{abstract}


% 英文摘要  (Abstract)
% 英文摘要内容应与中文摘要基本对应,符合英语语法,语句通顺。
% 关键词按相应专业的标准术语写出,每个关键词组的第一个字母大写,其余为
% 小写,每一关键词之间用分号分开,最后一个关键词后不打标点符号
\begin{englishabstract}
The main work of the thesis is ...

\englishkeywords{Kewword1; Keyword2; Keyword3}
\end{englishabstract}


  % 符号说明
  % 如论文中不需要添加符号说明, 则将下一行注释
  % !TEX encoding = UTF-8
% 符号说明页
% 解释论文中所用符号表示的意义、单位或量纲等.
% 符号说明主要为帮助阅读论文, 如论文中没有采用符号则该部分略去.
\begin{symbols}


\section*{微分几何常用符号说明}



\begin{table}[htbp]
    \centering
    \caption{微分几何常用符号表}
    \label{tab:symbols}
    \begin{tabularx}{\textwidth}{l>{\raggedright\arraybackslash}Xl}
        \toprule
        \textbf{符号} & \textbf{含义} & \textbf{备注} \\
        \midrule
        $M$, $N$ & 光滑流形 & 通常为有限维 \\
        $\mathcal{F}(M)$ & $M$ 上光滑函数环 & \\
        $T_p M$ & 流形 $M$ 在点 $p$ 处的切空间 & \\
        $T M$ & 流形 $M$ 的切丛 & \\
        $\mathfrak{X}(M)$ & $M$ 上的光滑向量场全体 & \\
        $\omega$, $\eta$ & 微分形式 & \\
        $\Omega^k(M)$ & $M$ 上的 $k$-次微分形式空间 & \\
        $d$ & 外微分算子 & $d: \Omega^k(M) \to \Omega^{k+1}(M)$ \\
        $\wedge$ & 外积(楔积) & \\
        $\mathcal{L}_X$ & 关于向量场 $X$ 的李导数 & \\
        $[X, Y]$ & 向量场 $X$ 与 $Y$ 的李括号 & \\
        $\nabla$ & 仿射联络(或协变导数) & \\
        $\nabla_X Y$ & 向量场 $Y$ 沿 $X$ 方向的协变导数 & \\
        $\Gamma_{ij}^k$ & Christoffel 符号 & 对于坐标基 $\partial_i$ \\
        $g$, $\langle \cdot, \cdot \rangle$ & 黎曼度量 & \\
        $R$ & 曲率张量 & $R(X,Y)Z = \nabla_X \nabla_Y Z - \nabla_Y \nabla_X Z - \nabla_{[X,Y]} Z$ \\
        $R_{ijk}^{\ \ \ l}$ & 曲率张量的坐标分量 & \\
        $\mathrm{Ric}$ & Ricci 曲率张量 & \\
        $S$ & 标量曲率 & \\
        $\exp_p$ & 在点 $p$ 处的指数映射 & \\
        $\mathrm{inj}(p)$ & 点 $p$ 处的单射半径 & \\
        $\mathrm{Hess}(f)$ & 函数 $f$ 的 Hessian & 在黎曼流形上 $\mathrm{Hess}(f) = \nabla^2 f$ \\
        $\Delta$ & Laplace–Beltrami 算子 & \\
        $\partial M$ & 流形 $M$ 的边界 & \\
        $\nu$ & 边界上的单位外法向量场 & \\
        $\mathbb{S}^n$ & $n$ 维球面 & \\
        $\mathbb{RP}^n$ & $n$ 维实射影空间 & \\
        $\mathbb{CP}^n$ & $n$ 维复射影空间 & \\
        \bottomrule
    \end{tabularx}
\end{table}

\end{symbols}



%%--------------------------------------------------------------------
%|===========|%
%| 正文部分        |%
%|===========|%
\mainmatter   % 正文部分开始

% 前言、正文、结语等

  %%输入第一章各节
  % !TEX encoding = UTF-8

%如果从第0章开始, 则修改计数器 chapter 为 -1
\setcounter{chapter}{-1}

\chapter{前言}

YZUthesis 模板最初完成于作者攻读博士学位期间, 基于 CASthesis\acknowledge{CASthesis 是中国科学院硕士、博士论文LaTeX模板, 作者吴凌云. 模板可在github上下载. \href{https://github.com/huashi1/CAS_Thesis_Latex} {https://github.com/huashi1/CAS\_Thesis\_Latex}.}, 其中导师签字页的设计, 则参考了加拿大达尔豪斯大学(Dalhousie University)的学位论文模板.

历经多年, 原模板已难以适配最新 TeXLive 系统的编译环境, 更新工作迫在眉睫. 本次修订基本遵循《扬州大学研究生学位论文格式规范(试行)》(校研院〔2020〕4号)(简称“论文格式规范文件”)的要求, 力求在格式上基本符合学校规定. 我们希望这一更新能为研究生们提供切实的便利, 使其从繁琐的排版调整工作中解脱出来, 将宝贵的时间与精力投入于论文写作与打磨中.



\section{学位论文的格式规范}

这里列举一些“论文格式规范文件”中关于论文格式规范需要注意的地方. 该文件规定了论文页面的具体设置\footnote{版芯要求: 左边距: 26mm, 右边距: 26mm, 上边距: 30mm, 下边距: 25mm. 页眉距边界20mm, 页脚距边界17.5mm.}. 学位论文由下列部分组成并按此顺序排列.
\begin{itemize}
  \item (1)封面;
  \item (2)英文扉页;
  \item (3)基金资助页;
  \item (4)论文原创性声明和版权使用授权书;
  \item (5)目录;
  \item (6)中文摘要及关键词;
  \item (7)英文摘要及关键词;
  \item (8)符号说明;
  \item (9)前言(绪言或绪论);
  \item (10)正文;
  \item (11)结语;
  \item (12)参考文献;
  \item (13)附录;
  \item (14)攻读学位期间取得的研究成果;
  \item (15)致谢.
\end{itemize}

不过本模板将参考文献放在附录的后面.








  % !TEX encoding = UTF-8


\section{论文写作注意事项}

用中文撰写数学论文或书籍时, 一般采取英文标点符号, 并且要在标点后加一个半角空格. 论文中如出现外国人名, 则建议要么全部使用中文(首次出现时可在其后加上原本国语言写法), 要么全部使用英文, 确保术语和专有名词的一致性.

  % !TEX encoding = UTF-8
\chapter{使用说明}

模板中所有文件均采用 \verb|UTF8| 编码存储, 方便跨平台编译. 如果使用的是 WinEdt, 则将\z{thesis.tex}或\z{ThesisForCheck.tex}设置为主文件(即打开(或用鼠标选中已打开的)\z{thesis.tex}文件或\z{ThesisForCheck.tex}文件, 点击菜单栏中的\z{Project\eq{\rt}Set Main File}将其设置为主文件). 然后点击编译按钮(比如 PDFLaTeX等)编译得到 pdf 文件. 注意\z{thesis.tex}和\z{ThesisForCheck.tex}共用其他所有文件. 一般需要编译多次, 如果有参考文献和索引, 还需运行 \verb|bibtex| 和 \verb|makeindex|. 下面以 \verb|latex| 编译为例, 依次执行下面的命令.

\begin{enumerate}

\item
快捷键\z{Ctrl+Shift+L}或命令行
\begin{verbatim}
latex thesis
\end{verbatim}

\item

快捷键\z{Ctrl+Shift+B}或命令行
\begin{verbatim}
bibtex thesis
\end{verbatim}

\item
快捷键\z{Ctrl+Shift+I}或命令行
\begin{verbatim}
makeindex thesis
\end{verbatim}


\item
两次快捷键\z{Ctrl+Shift+L}或命令行
\begin{verbatim}
latex thesis
\end{verbatim}

\item 
使用 \verb|dvipdfmx| 工具将 \verb|.dvi| 文件转换为 \verb|.pdf| 文件.

\begin{verbatim}
dvipdfmx thesis.dvi
\end{verbatim}
\end{enumerate}


为方便编译, 这里提供了 Windows 下的Powershell 脚本 \verb|build.ps1|, 可右键并点击“使用 Powershell 运行”或在控制台下运行. 也可以在控制台下运行.

在控制台运行 \verb|build.ps1| 会出现下面的提示:
\begin{verbatim}
安全警告
请只运行你信任的脚本。虽然来自 Internet 的脚本会有一定的用处,但此脚本可能会损坏你的计算机。如果你信任此脚本,请使用 Unblock-File
cmdlet 允许运行该脚本,而不显示此警告消息。是否要运行 D:\YZUthesis\build.ps1?
[D] 不运行(D)  [R] 运行一次(R)  [S] 暂停(S)  [?] 帮助 (默认值为“D”):
\end{verbatim}

此时输入 \verb|R| 或 \verb|r| 按回车即可. 此时默认使用 \verb|xelatex| 引擎先后编译 \verb|thesis.tex| 和 \verb|ThesisForCheck.tex|, 最后得到 \verb|thesis.pdf| 和 \verb|ThesisForCheck.pdf| 两个文件.

\verb|build.ps1| 也可以后面跟参数, 指定编译引擎以及要执行的动作. 如果跟 \verb|-Help| 或 \verb|-help| 或 \verb|-h| 则列出 \verb|build.ps1| 的帮助信息. 如下:


\begin{verbatim}
===========================================
YZUthesis 模板编译系统 (PowerShell 版本)
===========================================

用法: .\build.ps1 [-Action <命令>] [-Engine <引擎>] [-Help | -help | -h]

可用命令 (Action):

  all             - 编译所有内容
  cls             - 只生成 .cls 和 .cfg 文件
  doc             - 只生成模板文档
  thesis          - 编译所有主文档 (默认)
  check           - 只编译检查文档
  clean           - 清理临时文件
  distclean       - 完全清理

可用引擎 (Engine):

  pdflatex        - 使用 pdfLaTeX 编译
  xelatex         - 使用 XeLaTeX 编译 (默认)
  lualatex        - 使用 LuaLaTeX 编译
  latex           - 使用传统 LaTeX 编译,生成 DVI 再转 PDF

示例:

  .\build.ps1                            # 默认编译所有 (xelatex)
  .\build.ps1 -Engine pdflatex           # 使用 pdfLaTeX 编译所有
  .\build.ps1 -Action thesis -Engine lualatex
  .\build.ps1 -Action clean              # 清理临时文件
  .\build.ps1 -Help                      # 显示帮助信息

当前配置:
  默认引擎: xelatex
  主文档: thesis.tex, ThesisForCheck.tex
\end{verbatim}

\medskip





  %%输入第二章各节
  % !TEX encoding = UTF-8
\section{数学}

\subsection{定理等环境}

模板已经预设了 \verb|theorem|、\verb|lemma| 等定理环境, 详见 \verb|YZUthesis.dtx| 或由此生成的 \verb|YZUthesis.cls| 文件. 内容如下:

\begin{verbatim}
\newtheorem{theorem}{定理}[chapter]
\newtheorem{lemma}[theorem]{引理}
\newtheorem{corollary}[theorem]{推论}
\newtheorem{proposition}[theorem]{命题}
\newtheorem{definition}[theorem]{定义}
\newtheorem{example}[theorem]{例}
\newtheorem{remark}[theorem]{注}
\newtheorem{assumption}[theorem]{假设}
\newtheorem{axiom}[theorem]{公理}
\end{verbatim}

在 \verb|setup/cformat.tex| 文件中, 可以自己设定定理环境. 如:
\begin{verbatim}
\newtheorem{thm}{定理}[section]
\newtheorem{cor}[thm]{推论}
\newtheorem{lem}[thm]{引理}
\newtheorem{prop}[thm]{命题}
\theoremstyle{definition}
\newtheorem{defn}[thm]{定义}
\newtheorem{conj}{猜想~}
\newtheorem{exmp}{例~}
\newtheorem{exer}{习题~}
%\theoremstyle{remark}
\newtheorem{rem}[thm]{注}
\numberwithin{equation}{section}
\renewcommand{\proofname}{\bf 证明}
\end{verbatim}

这两种都可以使用.


\subsection{样例}

\begin{theorem}[Bernstein]
设 $f$ 在 $[a,b]$ 上任意阶可导, 且各阶导数非负. 则当 $x,x_0\in(a,b)$, 且 $|x-x_0| < b-x_0$ 时,
\[
f(x)=\sum_{n=0}^{\infty}\frac{f^{(n)(x_0)}}{n!}(x-x_0)^n\ .
\]
\end{theorem}
\begin{proof}
见\href{http://www.atzjg.net/admin/do/view_answers.php?qid=3515}{问题3515}.
\end{proof}


\begin{thm}[Dini] 设 $\{g_n(x)\}_{n=1}^{\infty}$ 是定义在 $[a,b]$ 上的非负连续函数列, 若对每个 $x\in[a,b]$, $\{g_n(x)\}$ 单调递减趋于 $0$, 则 $g_n\rightrightarrows 0$.
\end{thm}
\begin{proof}
[分析] 要证明 $\{g_n\}$ 一致收敛到 $0$, 即任给 $\varepsilon > 0$, 要找到仅依赖于 $\varepsilon$ 的正整数 $N$, 使得当 $n > N$ 时, 有 $|g_n(x)-0| < \varepsilon$.

由于 $g_n$ 非负, 故上面的不等式等价于 $g_n(x) < \varepsilon$.

对于这个不等式, 所给的条件无法用于放缩进而证明小于任给的 $\varepsilon$. 条件点点收敛 $g_n(x)\rightarrow 0$ ($n\rightarrow\infty$) 得到的 $N$ 是依赖于 $\varepsilon$ 和 $x$ 的. 因此转而考虑其他方法, 比如反证法.

$\{g_n\}$ 不一致收敛到 $0$ 等价于存在 $\varepsilon >0$, 对任意的 $N$, 存在 $n > N$, 使得 $g_n(x)\geqslant\varepsilon$ 对某个 $x\in[a,b]$ 成立.

为此, 考虑集合

\[
A_n=\{x\in[a,b]\mid g_n(x)\geqslant\varepsilon\}.
\]

我们只需要证明从某个 $N$ 往后的集合 $A_n$ 都是空集.

由于 $\{g_n(x)\}$ 关于 $n$ 是单调递减的, 故 $g_{n+1}(x)\leqslant g_n(x)$. 因此, 若 $x\in A_{n+1}$, 则 $x\in A_n$. 于是我们有

\[
A_1\supset A_2\supset\cdots\supset A_n\supset A_{n+1}\supset\cdots\ .\tag{*}
\]

我们只要证明存在某个 $A_n=\emptyset$. 此时又使用反证法. 加上对于任意的 $n\geqslant 1$, $A_n\neq\emptyset$.

即存在 $x_n\in A_n$, $n=1,2,\ldots$.  由于 $x_n\in[a,b]$, 故存在收敛子列 $x_{n_k}$, 其极限(记为 $x_0$)仍属于 $[a,b]$. 注意到 $n_k\geqslant k$. 故

\[A_k\supset A_{n_k}\supset\{x_{n_k}, x_{n_{k+1}},\ldots\}.\]

又 $g_n(x)$ 在 $[a,b]$ 上连续, 故对于收敛到 $x_0$ 的点列 $\{x_{n_k}\}$, 有

\[
\lim_{k\rightarrow\infty}g_n(x_{n_k})=g_n(x_0).
\]

而 $g_n(x_{n_k})\geqslant\varepsilon$, 故由极限的保号性知道 $g_n(x_0)\geqslant\varepsilon$. 这与条件 $g_n(x)$ 对每个 $x$ 关于 $n$ 递减趋于 $0$ 矛盾. 因此 $(*)$ 中存在某个 $A_{N}$ 是空集. 从而得证.
\end{proof}


Dini 定理可以用于函数项级数一致收敛的判断.

\begin{corollary}
设 $\sum\limits_{n=1}^{\infty}f_n(x)$ 在 $[a,b]$ 上收敛到 $f(x)$, 且 $f_n(x)$ 非负连续, 则 $\sum\limits_{n=1}^{\infty}f_n(x)$ 在 $[a,b]$ 上一致收敛到 $f(x)$.
\end{corollary}


以上定理参见 \cite{Mei2013B}.


\subsection{行内公式}

\begin{verbatim}
    由于中英文之间需要适当的间距(一般使用~来调节), 如果使用 $ $, 则需要敲入 ~$ $~ . 因此建议改用 \eq{ }. `\eq{}' 的定义为:
    \newcommand{\eq}[1]{~$#1$~}
    详见 cformat.tex, 其中还定义了其他常用的命令.

试比较
    公式$f(x)$与前后中文之间无适当的间距, 但\eq{f(x)}看起来更舒服.
效果如下:
\end{verbatim}

  公式$f(x)$与前后中文之间无适当的间距, 但\eq{f(x)}看起来更舒服.

注意早就有更好的办法解决这个问题, 请网上搜索这个问题. 例如 XeTeX
0.997中文间距调整宏包\z{zhspacing}.

\begin{verbatim}
使用 \eq{} 有个缺点, 就是若正好标点符号出现在公式后面, 如
\eq{\sqrt{1+x}}. 则句点 `.' 可能会出现在行首. 因此,
我建议在这种情况下输入\eq{\sqrt{1+x}.}
\end{verbatim}

标点符号有可能出现在行首,
如\eq{\sqrt{1+x}}.\eq{x},\eq{f(x)=\sin(x)\cos(x)}.\eq{\sqrt{1+x}}.\eq{x},\eq{f(x)=\sin(x)\cos(x)}.
\eq{\sqrt{1+x}}.\eq{x},\eq{f(x)=\sin(x)\cos(x)}.\eq{\sqrt{1+x}}.\eq{x},\eq{f(x)=\sin(x)\cos(x)}.


\eq{\mathpzc{f(x)}}
\[
\mathpzc{ABCDEFGHIJKLMNOPQRSTUVWXYZ}
\]
\[
\mathpzc{abcdefghijklmnopqrstuvwxyz}
\]
\[
\mathpzc{1234567890}
\]

  % !TEX encoding = UTF-8
\section{注意事项}

\begin{itemize}
\item
旧版本 \verb|YZUthesis V1.0| 仅在中文\z{CTeX}套装(v2.4.6)下测试通过, 不再建议使用.

\item
新版本 \verb|YZUthesis V2.0| 在 TeXLive 2023 中测试通过. 建议安装 TeXLive2023 或更高版本.

\item
\z{fig}文件夹中的图片在学位论文封面中要用到, 不要变动.

\item
建议将自己要用到的图片文件放在\z{img}文件夹中.
\end{itemize}

  % !TEX encoding = UTF-8
\section{参考文献}

可以自己设定参考文献的样式, 例如仿照\z{sample-bbl.txt}直接修改\z{thesis.bbl,}
注意文件\z{thesis.bbl}是由 \verb|bibtex thesis| 生成的. 最好备份一下修改好的新的\z{.bbl}文件, 比如重命名为 \verb|thesis-bbl.tex|, 当需要使用时可以再把内容拷贝到\z{thesis.bbl}中. \z{sample-bbl.txt}中的格式也可以根据需要自己修改.


\begin{remark}
注意当使用\z{Ctrl+Shift+X}时, 会自动运行\z{bibtex}命令,
从而可能会修改你可能已经重新修改好的\z{thesis.bbl}文件.
因此别忘了备份你自己的\z{.bbl}文件.
\end{remark}


\noindent\textbf{参考文献示例}

引用参考文献\cite{Chavel1993B},\cite{Mei2025B},

\cite{Ivanov1987A},

\cite{MLZ1986B},

\cite{Ballmann-Gromov-Schroeder1985B}.


  %%输入第三章各节
  % !TEX encoding = UTF-8
\chapter{一些小工具}

\z{tools}目录下有一些小工具.

\section{replace.exe}

\noindent\textbf{功能:}\
将\z{.tex}文件中的\$\$符号替换为\eq{\backslash\textrm{eq}\{\}.}

例如: 双击\z{replace.exe,}
输入\z{sample.tex,}它将输出\z{out.sample.tex.}

  % !TEX encoding = UTF-8
\section{creat-chap-folder.exe}

\noindent\textbf{功能:}\ 按照你的需要自动创建\z{chap}文件夹.

如果有很多章节, 那么建立\z{chap}文件夹将是一件很麻烦的事情.

双击\z{creat-chap-folder.exe,}输入所需的章节数, 程序将自动为你
生成\z{chap}文件夹.




  % !TEX encoding = UTF-8
\section{split.exe}

程序名称: split.exe

(配合扬州大学博硕士论文模板\z{yzuthesis}使用)

作者: H. Xu

完成时间: 2008-07-06

用途: 本程序是专门为论文上传而制作的, 可以一次性生成所需要的
      封面.pdf, 摘要.pdf, 目录.pdf, 正文.pdf

使用说明: 将\z{split.exe}拷贝到与\z{thesis.tex}所在目录下 (即位于
\z{yzuthesis}目录中), 然后双击运行.

但是有几项需要注意.

注意事项: 1. 所生成的\z{pdf}文件放在\z{upload}目录中,
             \z{upload}目录与\z{yzuthesis}目录同级.
             所生成的\z{pdf}文件中的中文可以复制拷贝, 因为本程序
             中使用的是\z{dvipdfmx}命令.

          2. 由于目前还没有设计专门的\z{bibtex}格式文件, 因此
             \z{thesis.bbl}是手工输入的. 为安全起见, 请将您自己
             写的\z{thesis.bbl}文件另存为\z{thesis-bbl.bblbak}, 注意
             本程序只认\z{thesis-bbl.bblbak}这个文件名 (这是我自
             定的, 有点罗嗦, 不好意思), 使用其它
             文件名将导致程序找不到该文件而无法完成任务.

          3. 摘要中如果有引用, 如
\begin{center}
\begin{verbatim}
\cite{Gromov1982}
\end{verbatim}
\end{center}
             请手工改成\z{``[Gromov1982]''}等等. 否则会出现问号 ? .

          4. 本程序自动清除所产生的临时文件, 并且不会改动原来
             的\z{.tex}文件.

          5. 本程序运行时间可能比较长, 一般需要2分钟左右.


%%--------------------------------------------------------------------
% 结语 peroration
  % !TEX encoding = UTF-8
\safechapter*{结语}

本文聚焦于...问题, 主要贡献在于: ... . 本研究 ... .



%%--------------------------------------------------------------------
% 附录
  \appendix

  % !TEX encoding = UTF-8
%在附录中使用 \safechapter 代替 \chapter
\safechapter{开发历史}

\section{简要历史}

\begin{itemize}
  \item 2007.3 YZUthesis V1.0
  \item 2007--2008 加入了一些小工具: replace.exe, create-chap-folder.exe,
split.exe
  \item 2008.6 YZUthesis V1.1, 按扬大要求加入版权说明页(最后一页), 没有大的改动.
  \item 2026.1 YZUthesis V2.0, 完成于扬州大学数学学院.
\end{itemize}





\section{欢迎}

因水平所限, 模板中难免有错误之处. 欢迎大家一起改进, 或者给出建议. 感谢支持, 祝使用愉快!


%正文中不能出现名字, 因为 ThesisForCheck.tex 生成抽检用的 pdf文件中不允许出现名字.
%\begin{flushright}
%    H. Xu
%\end{flushright}

  % !TEX encoding = UTF-8
\safechapter{更新}

\section{模板文件组织}


%%--------------------------------------------------------------------
%|======================================================|%
%| 附件部分, 包括参考文献、索引、发表文章目录、个人简历、致谢           |%
%|======================================================|%
\backmatter

  % 参考文献
  % 使用 BibTeX
  %input "bib/xbib.bib"

  % 参考文献部分
  \bibliographystyle{plain} % 选择参考文献样式
  \bibliography{bib/xbib} % 指定bib文件(不要写.bib扩展名)
  %\nocite{*}%%使用\nocite{*}将把所有未引用的参考文献也列出

  %%\bibliographystyle{}中的选项, 这里我们使用最普通的 plain.
  %%其余的有
  %%alpha 按字母顺序排列, 标签由作者姓名和出版年份组成.
  %%plain 按字母顺序排列, 标签为数字.
  %%unsrt 类似于 plain, 但按正文中引用先后排序.
  %%abbrv 类似于 alpha, but more compact labels.
  %%thubib.bst 属于清华大学模板.
  %%其他参考文献的样式文件有 amsalpha.bst, amsplain.bst 等等.
%%--------------------------------------------------------------------

  % 打印索引
  \printindex

  % 发表文章目录. 如不需要请注释掉
  % !TEX encoding = UTF-8
% 攻读学位期间取得的研究成果

\begin{pub}

\section{学术论文}

\begin{publications}{99}
\item 论文1: Title 1, Journal, 2020.
\item 论文2: Title 2, Conference, 2021.
\item 论文3: Title 3, Journal, 2022.
\item 论文4: Title 4, Journal, 2023.
\end{publications}


\section{科研项目}

\begin{publist}{}
\item \textbf{主持}, XX项目(项目编号:XXX), ``项目标题'', 202?.01-202?.12, 经费:?
\item \textbf{主要参与人}, 国家自然科学基金面上项目(项目编号:56781234), ``项目题目'', 202?.01-202?.12, 经费:??万元. (排名第?)
\item \textbf{参与}, XX青年项目(项目编号:BK???????), ``项目标题'', 202?.06-202?.06, 经费:??万元. 
\end{publist}

\section{学术奖励}

\begin{publist}{}
\item 2023年, “xxx奖”(校级, 排名第一)
\item 2022年, “xxx奖”(省级, 排名第一)
\item 2021年, “xxx奖”(国家级, 排名第一)
\end{publist}


\end{pub}





  % 个人简历. 如不需要请注释掉
  % !TEX encoding = UTF-8
% 简历

\begin{resume}
\section{教育背景}
\begin{resumelist}{教育状况}
\item 20??年~9月至~20??年~7月, 某某大学, 某某专业, 学士.
\item 20??年~9月至~20??年~7月, 某某大学, 某某专业, 硕士.
\item 20??年~9月至~20??年~7月, 某某大学, 某某专业, 博士. 
\end{resumelist}

\section{工作经历}
\begin{resumelist}{工作经历}
\item 无
\end{resumelist}

\end{resume}


  % 致谢. 如不需要请注释掉
  % !TEX encoding = UTF-8
% 致谢

\begin{thanks}
\kaishu

本模板是在\z{CASthesis}宏包的基础上实现的, 并据此作了必要的定制化调整, 以适应具体的格式要求. 谨此向宏包原作者表示诚挚感谢. 此外, DeepSeek在模板后续维护与优化过程中提供了高效辅助, 在此一并致谢.


\vskip 18pt

\hfill{H. Xu\qquad\qquad}

\hfill{二零二六年一月\quad}

\end{thanks}



%%--------------------------------------------------------------------

\end{document}

%%<*>--<*>--<*>--<*>--<*>--<*>--<*>--<*>--<*>--<*>--<*>--<*>--<*>--<*>
